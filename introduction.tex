\chapter*{Introduction générale}
\addcontentsline{toc}{chapter}{Introduction générale} % to include the introduction to the table of content
\markboth{Introduction générale}{} %To redefine the section page head

{\vspace{1cm \large Le marché de l'assurance est en pleine mutation : Nouvelles attentes de l’assuré, plus digital, plus volatil et mieux informé,  durcissement du cadre réglementaire, concurrence de plus en plus forte. Pour rester compétitifs, les assureurs doivent sans cesse se réinventer, innover et simplifier leur offre, leur distribution et leur relation assuré.\\
Dans ce contexte, optimiser les processus de gestion de l’assurance et accélérer la distribution directe et indirecte  sont deux chantiers de transformation digitale à prioriser car ils donneront aux assureurs un temps d'avance sur le marché, sur leurs concurrents et sur leurs objectifs de performance.\\
c'est pour cela les assurtech sont apparus. En effet, ce sont des entreprises exerçant dans le secteur de l’assurance. Elles s’appuient sur les nouvelles technologies pour introduire des innovations qui conduisent inéluctablement à l’éclosion de nouveaux modèles économiques, de nouveaux processus, de nouveaux produits.
Ces transformations profondes ont la capacité de modifier les comportements de tous les acteurs du marché : assurés, intermédiaires d’assurance, assureurs, réassureurs.
Les solutions offertes par les assurtech sont plus nombreuses en assurance non vie. Les investissements réalisés y sont plus importants du fait de la forte présence de la télémétrie dans les assurances automobile, santé et habitation.\\
Les améliorations qu’elles tentent d’apporter dans ce secteur visent à développer et enrichir les services offerts aux assurés tout en réduisant les coûts.}}